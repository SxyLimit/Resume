% !TEX TS-program = xelatex
% !TEX encoding = UTF-8 Unicode
% !Mode:: "TeX:UTF-8"

\documentclass{resume}
\usepackage{zh_CN-Adobefonts_external} % Simplified Chinese Support using external fonts (./fonts/zh_CN-Adobe/)
%\usepackage{zh_CN-Adobefonts_internal} % Simplified Chinese Support using system fonts
\usepackage{linespacing_fix} % disable extra space before next section
\usepackage{cite}

% ---------------- Template Config ----------------
\setstretch{0.93}
\renewcommand{\ResumeWatermarkText}{简历模板}
\renewcommand{\CompanyIconHeight}{1.725em}
\renewcommand{\OrgTitleWidth}{5.4cm}

\begin{document}
\pagenumbering{gobble} % suppress displaying page number
\EnableResumeWatermark

{
    \noindent
    \begin{minipage}[b]{0.75\textwidth}
        {\LARGE \textbf{简历模板展示(中文)}} \\[5pt]
        \contactline{\faPhone}{电话}{(+1) 617-555-0142} \\
        \contactline{\faEnvelope}{邮箱}{resume.template@example.com} \\
        \contactline{\faGithub}{主页}{\href{https://github.com/resume-template}{\underline{github.com/resume-template}}}
    \end{minipage}%
    \hfill
    \begin{minipage}[b]{0.2\textwidth}
        \raggedleft
        \includegraphics[height=2.8cm, keepaspectratio]{IDphoto.png}
    \end{minipage}
    \vspace{0.2em}
}

\section{\faGraduationCap\ 教育经历}
\edusubsectionwithicon[pku.png]{\textbf{\makebox[5cm][l]{北京大学}} \textit{计算机科学与技术}}{\textit{本科} \hfill 2017.9 - 2021.6}
\edusubsectionwithicon[mit.svg]{\textbf{\makebox[7.2cm][l]{麻省理工学院(MIT)}} \textit{计算机科学}}{\textit{博士} \hfill 2021.9 - 2026.6}
\begin{itemize}
    \item \textbf{研究方向}:大语言模型、推理增强、智能体系统与多模态学习。
    \item \textbf{代表课程}:机器学习、概率图模型、分布式系统、高级算法、统计推断与优化方法。
    \item \textbf{学术活动}:担任多个国际顶会审稿人并参与跨校联合研究项目。
\end{itemize}

\section{\faBook\ 论文发表}
\begin{itemize}
    \item \textbf{Attention Is All You Need}(arXiv:1706.03762,2017)
    \item \textbf{作者}:Ashish Vaswani, Noam Shazeer, Niki Parmar, Jakob Uszkoreit, Llion Jones, Aidan N. Gomez, Lukasz Kaiser, Illia Polosukhin
    \item \textbf{核心贡献}:提出仅依赖注意力机制的 Transformer 架构,完全去除 RNN/CNN,显著提升并行训练效率与建模能力。
    \item \textbf{关键结果}:WMT14 英德翻译 BLEU 28.4(较当时最优提升 2+);英法翻译 BLEU 41.8,达到单模型 SOTA。
\end{itemize}

\section{\faTrophy\ 荣誉奖项}
\begin{itemize}
    \item 诺贝尔物理学奖(Nobel Prize in Physics) \hfill \textbf{2024}
    \item 诺贝尔化学奖(Nobel Prize in Chemistry) \hfill \textbf{2023}
    \item 诺贝尔经济学奖(Nobel Memorial Prize in Economic Sciences) \hfill \textbf{2022}
\end{itemize}

\section{\faCogs\ 专业技能}
\begin{itemize}[parsep=0.2ex]
    \item \textbf{编程语言}:Python、C++、Rust、SQL。
    \item \textbf{模型与系统}:熟悉 PyTorch、JAX、分布式训练、RAG、Agent 工作流与在线推理优化。
    \item \textbf{工程能力}:能够从研究原型快速落地到可观测、可评测、可灰度发布的生产系统。
\end{itemize}

\section{\faBriefcase\ 实习经历}
\orgsubsection{Openai.ico}{https://openai.com/}{OpenAI}{研究工程实习生}{2025.6 - 2025.9}
\begin{itemize}
    \item \textbf{STAR 法则定义}:STAR 是结构化表达经历的方法,\textbf{S(Situation)} 指背景场景,\textbf{T(Task)} 指目标任务,\textbf{A(Action)} 指关键行动,\textbf{R(Result)} 指最终结果。
    \item \textbf{写作原则}:先交代背景与目标,再描述本人采取的核心动作,最后给出量化结果(如准确率、时延、成本、效率提升),避免只罗列“做了什么”。
\end{itemize}

\orgsubsection{gemini}{https://deepmind.google/technologies/gemini/}{Google DeepMind (Gemini)}{核心研究实习生(Gemini 3.1 Pro)}{2025.10 - 2026.2}
\begin{itemize}
    \item \textbf{S(Situation)}:团队需要在固定算力预算内提升 Gemini 3.1 Pro 的综合能力并保证训练稳定性。 \textbf{T(Task)}:负责 Gemini 3.1 Pro 模型训练方案的设计、执行与迭代。 \textbf{A(Action)}:持续优化数据配比、训练调度与超参数设置,按阶段跟踪收敛曲线和核心评测表现,针对异常批次快速回滚并重训。 \textbf{R(Result)}:模型收敛更稳定、训练效率显著提升,在数学推理、代码与长上下文等多项指标达到当期 SOTA。
\end{itemize}

\hrule
\begin{itemize}[itemindent=-0.50cm]
  \item[] \inlineicon{github.svg}\textbf{Github}:\href{https://github.com/resume-template}{\underline{resume-template}},
          \inlineicon{wechat}\textbf{Wechat}:ResumeTemplate
\end{itemize}
\end{document}
