\documentclass{resume}
\usepackage{zh_CN-Adobefonts_external}
\usepackage{linespacing_fix}
\usepackage{cite}

\setstretch{0.95}
\renewcommand{\ResumeWatermarkText}{个人信息保密}
\renewcommand{\ResumeWatermarkFontSize}{60}
\renewcommand{\ResumeWatermarkAngle}{32}
\renewcommand{\ResumeWatermarkGray}{0.9}
\renewcommand{\CompanyIconDir}{icon/myresume}
\renewcommand{\CompanyIconHeight}{1.6em}
\renewcommand{\OrgTitleWidth}{4.5cm}
\renewcommand{\ContactLabelSep}{:}

\begin{document}
\pagenumbering{gobble}
\EnableResumeWatermark

{
    \noindent
    \begin{minipage}[b]{0.75\textwidth}
        {\LARGE \textbf{斯修远 (Xiuyuan Si)}} \\[5pt]
        \contactline{\faPhone}{电话}{(+86) 159-6755-3933} \\
        \contactline{\faEnvelope}{邮箱}{2303504461@qq.com} \\
        \contactline{\faWeixin}{微信}{Lirnit(可通过手机号 159-6755-3933 添加)}
    \end{minipage}%
    \hfill
    \begin{minipage}[b]{0.2\textwidth}
        \raggedleft
        \includegraphics[height=2.8cm, keepaspectratio]{myresume/IDphoto-myresume.jpg}
    \end{minipage}
    \vspace{0.2em}
}

\section{\faGraduationCap\ 教育经历}
\edusubsectionwithicon[bjtu.jpeg]{\textbf{\makebox[5cm][l]{北京交通大学}} \textit{信息与计算科学} }{\textit{本科} \hfill 2023.9 - 2027.6(预计)}
\begin{itemize}
    % 可选:\item \textbf{GPA}:本科阶段:3.4/4.0 (排名前 32\%)
    \item \textbf{竞赛经历}:
        \begin{itemize}
            \item 在第 49 届 ICPC World Finals 中获\textbf{第三名},东亚区\textbf{冠军};在第 48 届 ICPC World Finals 中获\textbf{第七名};
            \item 大学期间参加十余次 ICPC / CCPC 区域赛与决赛,均获金牌,多次排名前五,取得过亚军和季军的成绩。
        \end{itemize}
    \item \textbf{主修课程}:数学分析、高等代数、数据结构与算法、数值计算、运筹学、概率论、数理统计、实变函数等。
    \item \textbf{校园经历}:校算法俱乐部会长。组织多次校内程序设计竞赛与讲座,累计超千人参与;曾任杭电“钉耙编程”联赛出题人,近千支各大高校队伍参赛。
\end{itemize}

\section{\faTrophy\ 荣誉奖项}
\begin{itemize}
    \item 第 49 届国际大学生程序设计竞赛全球总决赛 ICPC World Finals\  \hfill \textbf{金牌(全球第三名)}
    \item 第 48 届国际大学生程序设计竞赛全球总决赛 ICPC World Finals\  \hfill \textbf{银牌(全球第七名)}
    \item 2023 - 2025 年 ICPC / CCPC 区域赛与决赛\  \hfill \textbf{金牌共计十四次}
    \item 2021 年全国青少年信息学奥林匹克竞赛冬令营(NOI WC)\ \hfill \textbf{金牌}
    % 可选:\item 2020、2021 年全国青少年信息学奥林匹克联赛\ \hfill \textbf{一等奖}
\end{itemize}

\section{\faCogs\ 专业技能}
\begin{itemize}[parsep=0.2ex]
    \item \textbf{编程语言和框架}:熟悉 C、C++ 和 Python。
    \item \textbf{算法能力}:熟练掌握经典算法(动态规划、贪心算法、图论算法)和高级数据结构(线段树、树状数组、平衡树)。在多次竞赛中独立解决仅有个位数队伍通过的难题,展现出卓越的算法应用能力。
    \item \textbf{数学与逻辑思维}:具备扎实的数理基础,熟悉组合数学、数论、概率论、数理统计等。具备强大的逻辑思维和问题分析能力,能够迅速理解复杂问题并制定高效解决方案。
\end{itemize}

\section{\faCode\ 项目经历}
\orgsubsection[11.5cm]{huawei.svg}{https://competition.huaweicloud.com/information/1000042168/introduction}{华为算法精英实战营——高性能动态内存管理算法}{\textbf{第五名}}{2025.1 - 2025.3}
\begin{itemize}
    \item 面向真实内存回收场景,设计支持频繁增删边与动态 Root 集调整的实时可达性判定算法,融合正向图整体删除与反向图局部删除策略,将更新复杂度控制在近线性于受影响子图规模;采用 \textbf{vector 有序邻接表}替代链表结构提升缓存命中率,使运行速度提升约 \textbf{20\%};同时设计动态算力调度函数自适应分配计算资源,使加权准确率评分提升约 \textbf{15\%},最终在全国近 500 支队伍中获得第 5 名。
    \item \href{https://github.com/SxyLimit/HuaweiChallenge-22nd}{\underline{代码链接}}
\end{itemize}
% 可选(同项目的无图标写法):
% \projectsubsection{\href{https://competition.huaweicloud.com/information/1000042168/introduction}{\underline{华为算法精英实战营——高性能动态内存管理算法}}}{\textbf{第五名}}{2025.1 - 2025.3}
% \begin{itemize}
%     \item 针对真实内存回收场景下的高性能动态内存管理需求,设计并实现能够在频繁增删边和动态调整 Root 集合下实时判断节点可达性的算法。面对数据分布源于真实系统特征导致常规算法难以兼顾性能与稳定性的挑战,创新性地融合了正向图整体删除与反向图局部删除的双策略,使用动态数组维护有序邻接表以提升缓存命中率,并设计了动态算力调度函数智能调整不同阶段的计算资源消耗。最终的解决方案在全国近五百支队伍中获得第五名,证明了算法在真实场景下的卓越性能。
%     \item \href{https://github.com/SxyLimit/HuaweiChallenge-22nd}{\underline{代码链接}}
% \end{itemize}

\section{\faBriefcase\ 实习经历}

% 可选新增经历:
% \orgsubsection{metabit-trading.jpg}{https://www.metabit-trading.com/}{乾象投资}{量化研究实习生}{2026.3 - 至今}
% \begin{itemize}
%     \item xxx
% \end{itemize}

\orgsubsection{bytedance.png}{https://www.bytedance.com/}{字节跳动}{Agent 研发实习生 | 产品研发和工程架构 - 创新能力}{2025.9 - 2025.12}
\begin{itemize}
    \item 从零设计并实现新一代业务 Agent 的核心记忆与反思迭代系统,提出\textbf{五层经验记忆架构(临时 / 活跃 / 聚合 / 归档 / 墓碑)},支持经验验证、晋升、聚合、回溯与失效管理,实现经验持续沉淀与稳定演化。
    \item 将「出错 $\rightarrow$ 原因分析 $\rightarrow$ 解决方案」抽象为结构化经验单元,新经验在临时层验证后晋升为规则,并在聚合层提炼为知识更新建议,对原始文档进行最小改写与规则固化,提升 Agent 的错误规避与自修复能力。
    \item 打通“\textbf{经验提取—知识回写}”工程闭环:从会话日志中抽取带证据的候选经验,经去重过滤后写入临时层,并基于聚合层生成更新建议自动回写,使 Agent 在后续任务中持续优化表现,增强复杂业务场景下的稳定性与泛化能力。
\end{itemize}


\orgsubsection{antgroup.png}{https://www.antgroup.com/}{蚂蚁集团}{算法研发实习生 | CTO - 基础智能技术部}{2025.5 - 2025.8}
\begin{itemize}
    \item 构建面向 Online Judge 的\textbf{题目格式自动修复流水线},基于 Qwen 设计高鲁棒提示模板与预处理流程,实现结构对齐与格式修复自动化,成功处理题目数约三万,提升人工标注效率约 \textbf{30\%},通过验收标准的题目占比超过 \textbf{95\%}。
    \item 构建\textbf{代码修复 SFT 数据体系}以提升模型代码能力。围绕 Base Model 在代码修复任务上的不足,设计多种数据构造方案(包括基于 AST 构造变异代码与利用 LLM 分析错误生成样本),通过错误原因约束筛选高质量数据并重构评测方式,构建包含详细 traceback 信息的大规模 SFT 数据集,使微调模型在获得错误反馈后具备\textbf{基于 traceback 的针对性修复能力},在原本无法一次通过的题目中约 \textbf{10\%} 可完成修复。
\end{itemize}

\hrule
\begin{itemize}[itemindent=-0.50cm]
    \item[] \inlineicon{github.svg}\textbf{GitHub}:\href{https://github.com/SxyLimit}{\underline{SxyLimit}},
            \inlineicon{cnblogs}\textbf{Cnblogs}:\href{https://www.cnblogs.com/Sxy\_Limit}{\underline{Sxy\_Limit}}
\end{itemize}
\end{document}
